\section{Discussions}

From an operational perspective, the results highlight the relevance of lightweight learning-based restoration models for both ground-based and onboard processing. Although traditional restoration pipelines remain robust, their sequential structure and computational cost limit their suitability for resource-constrained environments. In contrast, the proposed EDSR based approach achieves comparable image quality, while significantly reducing processing time, making it better aligned with near-real-time operational constraints.

Beyond computational efficiency, the robustness of EDSR to variations in blur and noise levels constitutes a key advantage in practical scenarios, where imaging conditions may deviate from nominal specifications due to acquisition geometry, temporal effects, or sensor aging. These properties suggests that learning-based restoration can serve as a viable alternative to traditional pipelines, not only as a ground-based acceleration strategy but also as a candidate for onboard preprocessing, enabling earlier and more flexible exploitation of satellite imagery.

Finally, when coupled with onboard object detection tasks, EDSR revealed to significantly improve the performance of the AI model. In these application-driven scenarios, exact reconstruction of the ground truth is not strictly required, in contrast to traditional ground-based restoration objectives. Instead, the objective is to enhance task-relevant features while maintaining physical plausibility. The observed improvements therefore reinforce the suitability of learning-based restoration as a task-oriented preprocessing step for onboard AI applications in remote sensing.

%From an operational standpoint, the results highlight the relevance of lightweight learning-based restoration models for both ground-based and onboard processing. While traditional restoration pipelines remain robust, their sequential structure and computational cost limit their applicability in resource-constrained environments. In contrast, the proposed EDSR-based approach achieves comparable image quality while significantly reducing processing time, making it more compatible with near-real-time constraints.

%Beyond computational efficiency, the robustness of EDSR to variations in blur and noise levels is a key asset in operational contexts, where imaging conditions may deviate from nominal specifications due to acquisition geometry, temporal effects, or sensor aging. These properties suggest that learning-based restoration can serve as a practical alternative to traditional pipelines, not only as a ground-based accelerator but also as a candidate for onboard pre-processing, enabling earlier and more flexible exploitation of satellite imagery.

%Finally, when coupled with onboard object detection tasks, EDSR revealed to significantly improve the performance of the AI model. These use cases are particularly interesting, as we do not require a perfect reconstitution of the ground truth, unlike traditional on-ground restoration pipelines. Generalising on these findings highlight the effectiveness of the learning-based restoration approach as a preprocessing step for AI-based tasks in onboard remote sensing imagery.



%%The results suggest that learning-based image restoration can offer practical
%%advantages over traditional pipelines in terms of computational efficiency and
%%adaptability. While further validation is required to fully assess onboard
%%deployment constraints, the observed robustness and reduced processing time
%%indicate a promising direction for future operational systems.

